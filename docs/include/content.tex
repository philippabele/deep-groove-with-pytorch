\chapter{Einleitung} 
    \autocite[vgl.][S.1]{01} \\
    \begin{figure}[h]
        \centering
        \includegraphics{images/dhbw-logo.jpg}
        \caption{Meine Grafik}
        \label{fig:meine-grafik}
    \end{figure}
       

\chapter{Theoretische Grundlagen}

    \section{Exploratory Data Analysis} 
	
	%Hinführung
	
	\noindent Frederik Hardwig beschreibt die Explorative Datenanalyse als zwei Dinge. Eine Methode um Daten zu analysieren, um die Informationen aus Datensätzen sehr effizient zu extrahieren, und einem Gedankenansatz, einer Einstellung zur Datenanalyse. Je mehr man über einen Datensatz und seine Beschaffenheiten weis, desto effizienter können im Folgenden Theorien auf ihre Relevanz und Richtigkeit untersucht werden. Das explorative Arbeiten mit Daten erfordert vom Analyst Skepsis und Offenheit. Skepsis gegenüber zusammenfassenden Methodiken, die die Aussagen von Daten verfälschen können, und so die Aussagekraft der Theorien aus der Datenanalyse schwächen. Auch die Offenheit gegenüber unerwarteter Muster soll immer vorgebracht werden. Es sollten nicht eine zuvor entstandene These bewiesen, sondern neue Annahmen getroffen und Wissen erarbeitet werden. 
	
	\noident
	% Beispiel Datensatz mit beispielfrage
	%mit schönen Bildern und Grafen

\chapter{Applikation}


\chapter{Implementierung}


\chapter{Zusammenfassung} % Fazit