\chapter{Introduction} 
    \autocite[vgl.][S.1]{01} \\
    \begin{figure}[h]
        \centering
        \includegraphics{images/dhbw-logo.jpg}
        \caption{Meine Grafik}
        \label{fig:meine-grafik}
    \end{figure}
       

\chapter{theoretical Basis}

    \section{Exploratory Data Analysis} 
	
	%Hinführung
	
	\noindent Frederik Hardwig describes the exploratory data analysis as two things. The first is a method of how to analyze data. The more the analyst knows about the data set, the more efficient he can work with it to test the relevance and validity of his Hypotheses. The second one is the way of thinking about the data itself and data analysis. Skepticism and openness describes Hardwig as the key Factors of the exploratory Mindset. The Analyst should not work with tools that summarizes data because it could lead to wrong results and Hypotheses. He should also be open to patterns he did not looked for in the beginning and should be always aware that not expected patterns can lead to the most interesting results in data. The goal of exploratory data analysis is not to prove prior theses, it is to explore the data and find new theses and generate new knowledge.
	
	%Frederik Hardwig beschreibt die Explorative Datenanalyse als zwei Dinge. Eine Methode um Daten zu analysieren, um die Informationen aus Datensätzen sehr effizient zu extrahieren, und einem Gedankenansatz, einer Einstellung zur Datenanalyse. Je mehr man über einen Datensatz und seine Beschaffenheiten weis, desto effizienter können im Folgenden Theorien auf ihre Relevanz und Richtigkeit untersucht werden. Das explorative Arbeiten mit Daten erfordert vom Analyst Skepsis und Offenheit. Skepsis gegenüber zusammenfassenden Methodiken, die die Aussagen von Daten verfälschen können, und so die Aussagekraft der Theorien aus der Datenanalyse schwächen. Auch die Offenheit gegenüber unerwarteter Muster soll immer vorgebracht werden. Es sollten nicht eine zuvor entstandene These bewiesen, sondern neue Annahmen getroffen und Wissen erarbeitet werden. 
	
	% Beispiel Datensatz mit beispielfrage
	% mit schönen Bildern und Grafen

\chapter{Applikation}


\chapter{Implementation}


\chapter{Zusammenfassung} % Fazit